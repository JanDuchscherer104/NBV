\section{Related work}
\label{sec:related_work}
\vspace{-0.25em}

\noindent\textbf{NBV methods requiring prior knowledge.} Many existing approaches assume prior knowledge of the 3D scene, which limits applications in previously unexplored environments. Some approaches \cite{devrim2017reinforcement, sun2021learning, zhang2021continuous} directly exploit a pre-existing 3D model or its approximation. Other methods rely on 2D maps, such as estimating the height of buildings as a rough 3D model \cite{jing2016view} or obtaining a 2.5D model of the scene \cite{zhou2020offsite}.
For scenes lacking pre-existing information, often a drone fly-through along a default trajectory is made to obtain an initial coarse reconstruction \cite{roberts2017submodular,hepp2018plan3d}. In contrast, our method does not require any prior information about the scene; instead, we start with two adjacent views and sequentially choose the best next views until a specific termination criterion is met. 

\noindent\textbf{Optimal view selection with expensive scene representations.}
A related line of work addresses the problem of selecting an optimal subset of images for reconstruction by using complex scene representations. These representations are often built from existing dense capture. Although the selection algorithms themselves do not require dense pre-capture imagery as input, they rely on scene representations, such as dense point clouds, meshes, or pretrained radiance fields, that are themselves costly to obtain. Earlier approaches targeted optimal view selection for Multi-View Stereo reconstruction \citep{hornung2008image, furukawa2010towards}, while more recent work explored view selection for Neural Radiance Fields (NeRF) and its variants \citep{smith2022uncertainty, pan2022activenerf, lee2023so, jiang2023fisherrf}. In contrast, our approach does not assume access to a dense capture or an expensive prebuilt representation; we acquire images directly from the optimal viewpoints predicted by the NBV policy.

% \newpage
\noindent\textbf{Criteria for predicting NBV.} Previous approaches use proxy metrics like coverage \cite{maver1993occlusions, peralta2020next, drone_save, hepp2018learn, chen2024gennbv} or information gain \cite{lee2022uncertaintyguidedpolicyactive, Islerinformationgainvolumetric3d, POTTHAST2014148, jiang2023fisherrf} for predicting NBVs. State-of-the-art NBV techniques, GenNBV \cite{chen2024gennbv} and ScanRL \cite{peralta2020next}, use Reinforcement Learning (RL) with coverage-based reward functions. While coverage-based policies may lead to reasonable reconstruction quality, they fail to account for complex structures and occlusions, often resulting in holes in the reconstructed scene. On the other hand, the information gain criterion is image-based and often lacks 3D knowledge of the scene, restricting generalization. Our work goes beyond coverage by directly maximizing the reconstruction quality, which leads to significant improvement under resource constraints, as shown empirically. Rather than relying on RL \cite{peralta2020next,chen2024gennbv}, we design a greedy sequential policy, VIN-NBV, trained with imitation learning to help predict the reconstruction improvement of next views.

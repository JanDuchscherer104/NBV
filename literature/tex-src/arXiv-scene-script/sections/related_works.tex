\section{Related Works}
\label{section:related}

%% Table with Language
\begin{table*}[t]
\caption{Complete set of structured language commands designed for detailing architectural layouts and object bounding boxes. Supported data types can include \texttt{int, float, bool}. It is important to note that the language's extensibility allows for easy augmentation by introducing new commands like \texttt{make\_prim, make\_pillar}, or enhancing existing commands, such as incorporating \texttt{is\_double\_door (bool)}.}
\centering
\begin{ttfamily}
\scriptsize
\begin{tabular}{cccc}
make\_wall (\textcolor{teal}{int}) & make\_door (\textcolor{teal}{int}) & make\_window (\textcolor{teal}{int}) & make\_bbox (\textcolor{teal}{int}) 
%& Tokenized scene sequence
\tabularnewline
\hline 
\hline 
 % &  &  & 
 % \tabularnewline
\multicolumn{1}{c|}{%
\begin{tabular}{cc}
id & \textcolor{teal}{int}\tabularnewline
a\_x & \textcolor{blue}{float}\tabularnewline
a\_y & \textcolor{blue}{float}\tabularnewline
a\_z & \textcolor{blue}{float}\tabularnewline
b\_x & \textcolor{blue}{float}\tabularnewline
b\_y & \textcolor{blue}{float}\tabularnewline
b\_z & \textcolor{blue}{float}\tabularnewline
%thickness & \textcolor{blue}{float}\tabularnewline
height & \textcolor{blue}{float}\tabularnewline
\end{tabular}} & \multicolumn{1}{c|}{%
\begin{tabular}{cc}
id & \textcolor{teal}{int}\tabularnewline
wall0\_id & \textcolor{teal}{int}\tabularnewline
wall1\_id & \textcolor{teal}{int}\tabularnewline
position\_x & \textcolor{blue}{float}\tabularnewline
position\_y & \textcolor{blue}{float}\tabularnewline
position\_z & \textcolor{blue}{float}\tabularnewline
width & \textcolor{blue}{float}\tabularnewline
height & \textcolor{blue}{float}\tabularnewline
\end{tabular}} & \multicolumn{1}{c|}{%
\begin{tabular}{cc}
id & \textcolor{teal}{int}\tabularnewline
wall0\_id & \textcolor{teal}{int}\tabularnewline
wall1\_id & \textcolor{teal}{int}\tabularnewline
position\_x & \textcolor{blue}{float}\tabularnewline
position\_y & \textcolor{blue}{float}\tabularnewline
position\_z & \textcolor{blue}{float}\tabularnewline
width & \textcolor{blue}{float}\tabularnewline
height & \textcolor{blue}{float}\tabularnewline
\end{tabular}} & \multicolumn{1}{c}{%
\begin{tabular}{cc}
id & \textcolor{teal}{int}\tabularnewline
class & \textcolor{teal}{int}\tabularnewline
position\_x & \textcolor{blue}{float}\tabularnewline
position\_y & \textcolor{blue}{float}\tabularnewline
position\_z & \textcolor{blue}{float}\tabularnewline
angle\_z & \textcolor{blue}{float}\tabularnewline
scale\_x & \textcolor{blue}{float}\tabularnewline
scale\_y & \textcolor{blue}{float}\tabularnewline
scale\_z & \textcolor{blue}{float}\tabularnewline
\end{tabular}} \tabularnewline


 % &  &  & \tabularnewline

% \hline 
\end{tabular}

\end{ttfamily}
\label{table:commands_and_parameters}
\end{table*}




\subsection{Layout Estimation}
Layout estimation is an active research area, aiming to infer architectural elements. Scan2BIM~\cite{murali2017indoor} proposes heuristics for wall detection to produce 2D floorplans.
Ochmann et al.~\cite{ochmann2019automatic} formulate layout inference as an integer linear program using constraints on detected walls. 
Shortest path algorithms around birds-eye view (BEV) free space~\cite{cabral2014piecewise} and wall plus room instance segmentation~\cite{chen2019floor} have also been explored.


Furukawa et al.~\cite{furukawa2009reconstructing} utilize a \textit{Manhattan world}-based multi-view stereo algorithm~\cite{furukawa2009manhattan} to merge axis-aligned depth maps into a full 3D mesh of building interiors.
RoomNet~\cite{lee2017roomnet} predicts layout keypoints while assuming that a fixed set of Manhattan layouts can occur in a single image of a room. 
LayoutNet~\cite{zou2018layoutnet} improves on this by predicting keypoints and optimising the Manhattan room layout inferred from them.
Similarly, AtlantaNet~\cite{pintore2020atlantanet} predicts a BEV floor or ceiling shape and approximates the shape contour with a polygon resulting in an \textit{Atlanta world} prior. 
SceneCAD~\cite{avetisyan2020scenecad} uses a graph neural network to predict object-object and object-layout relationships to refine its layout prediction.

Our approach stands out by requiring neither heuristics nor explicitly defined prior knowledge about architectural scene structure.
In fact, our method demonstrates geometric understanding of the scene which emerges despite learning to predict the GT scene language as a sequence of text tokens.

\subsection{Geometric Sequence Modelling}

Recent works have explored transformers for generating objects as text-based sequences.
PolyGen~\cite{nash2020polygen} models 3D meshes as a sequence of vertices and faces. 
CAD-as-Language~\cite{ganin2021computer} represents 2D CAD sketches as a sequence of triplets in protobuf representation, followed by a sequence of constraints. 
Both SketchGen~\cite{para2021sketchgen} and SkexGen~\cite{xu2022skexgen} use transformers to generate sketches.
DeepSVG~\cite{carlier2020deepsvg} learns a transformer-based variational autoencoder (VAE) that is capable of generating and interpolating through 2D vector graphics images. 
DeepCAD~\cite{wu2021deepcad} proposes a low-level language and architecture  similarly to DeepSVG, but applies it to 3D CAD models instead of 2D vector graphics.
Our approach stands out by utilising \textit{high-level} commands, offering interpretability and semantic richness. Additionally, while low-level commands can represent arbitrarily complex geometries, they lead to prohibitively longer sequences when representing a full scene.



The closest work to ours is Pix2Seq~\cite{chen2021pix2seq}. Pix2Seq proposes a similar architecture to ours but experiments only with 2D object detection, thus requiring domain-specific augmentation strategies. Another closely related work is Point2Seq~\cite{xue2022point2seq} that trains a recurrent network for autoregressively regressing continuous 3D bounding box parameters. Interestingly, they find the autoregressive ordering of parameters outperforms current standards for object detection architectures, including anchors~\cite{girshick2015fast} and centers~\cite{yin2021center}.




\iffalse
%% Table with Language
\begin{table}[t]
\centering
\begin{ttfamily}
\scriptsize
\resizebox{\columnwidth}{!}{%
\begin{tabular}{cccc}
make\_wall (\textcolor{teal}{int}) & 
make\_door (\textcolor{teal}{int}) & 
make\_window (\textcolor{teal}{int}) &
make\_bbox (\textcolor{teal}{int})
\tabularnewline
\hline 
 &  &  \tabularnewline
\multicolumn{1}{c|}{%
\begin{tabular}{cc}
id & \textcolor{teal}{int}\tabularnewline
a\_x & \textcolor{blue}{float}\tabularnewline
a\_y & \textcolor{blue}{float}\tabularnewline
a\_z & \textcolor{blue}{float}\tabularnewline
b\_x & \textcolor{blue}{float}\tabularnewline
b\_y & \textcolor{blue}{float}\tabularnewline
b\_z & \textcolor{blue}{float}\tabularnewline
height & \textcolor{blue}{float}\tabularnewline
\end{tabular}} & \multicolumn{1}{c|}{%
\begin{tabular}{cc}
id & \textcolor{teal}{int}\tabularnewline
wall0\_id & \textcolor{teal}{int}\tabularnewline
wall1\_id & \textcolor{teal}{int}\tabularnewline
position\_x & \textcolor{blue}{float}\tabularnewline
position\_y & \textcolor{blue}{float}\tabularnewline
position\_z & \textcolor{blue}{float}\tabularnewline
width & \textcolor{blue}{float}\tabularnewline
height & \textcolor{blue}{float}\tabularnewline
\end{tabular}} & \multicolumn{1}{c|}{%
\begin{tabular}{cc}
id & \textcolor{teal}{int}\tabularnewline
wall0\_id & \textcolor{teal}{int}\tabularnewline
wall1\_id & \textcolor{teal}{int}\tabularnewline
position\_x & \textcolor{blue}{float}\tabularnewline
position\_y & \textcolor{blue}{float}\tabularnewline
position\_z & \textcolor{blue}{float}\tabularnewline
width & \textcolor{blue}{float}\tabularnewline
height & \textcolor{blue}{float}\tabularnewline
\end{tabular}} & \multicolumn{1}{c}{%
\begin{tabular}{cc}
id & \textcolor{teal}{int}\tabularnewline
class & \textcolor{teal}{int}\tabularnewline
position\_x & \textcolor{blue}{float}\tabularnewline
position\_y & \textcolor{blue}{float}\tabularnewline
position\_z & \textcolor{blue}{float}\tabularnewline
angle\_z & \textcolor{teal}{float}\tabularnewline
scale\_x & \textcolor{blue}{float}\tabularnewline
scale\_y & \textcolor{blue}{float}\tabularnewline
scale\_z & \textcolor{blue}{float}\tabularnewline
\end{tabular}} \tabularnewline
\hline
 &  & \tabularnewline
\end{tabular}
}
\end{ttfamily}
\caption{Our basic language command set for room layouts and object detection.}
\label{table:commands_and_parameters}
\end{table}
\fi


